\documentclass[11pt,a4paper]{article}

% Packages
\usepackage[utf8]{inputenc}
\usepackage{amsmath,amssymb,amsfonts}
\usepackage{graphicx}
\usepackage{hyperref}
\usepackage{bookmark}
\usepackage[margin=1in]{geometry}
\usepackage{xcolor}
\usepackage{cite}


%----------------------- Title --------------------------------------
\title{Constraining Beyond the Standard Model Physics with DESI Cosmological Data}
\author{[Authors TBD]}
\date{\today}

\begin{document}

\maketitle

\begin{abstract}
We explore constraints on various Beyond the Standard Model (BSM) scenarios using cosmological data from the Dark Energy Spectroscopic Instrument (DESI). We consider light thermal relics, axion-like particles, cannibal dark matter, decaying dark matter and other models, analyzing how DESI measurements of large-scale structure and baryon acoustic oscillations can probe these new physics scenarios.
\end{abstract}

%-------------------------------------------------------------------------
\section{Introduction}
DESI provides unprecedented precision in mapping the large-scale structure of the universe through spectroscopic observations of millions of galaxies and quasars. This data offers powerful constraints on cosmological parameters and can probe physics beyond the standard cosmological model ($\Lambda$CDM).

\subsection{DESI Overview}
% Add details about DESI capabilities, data releases, key observables

\subsection{BSM Signatures in Cosmology}
% How BSM physics affects cosmological observables

\section{Light Thermal Relics}

Light thermal relics contribute to the effective number of relativistic species $N_{\text{eff}}$ and can leave imprints on:
\begin{itemize}
    \item The cosmic microwave background (CMB)
    \item The matter power spectrum
    \item Baryon acoustic oscillations (BAO)
\end{itemize}

\subsection{Theoretical Framework}
% Add discussion of light relic freeze-out, effective number of species

\subsection{Observable Effects}
% How light relics modify growth of structure, BAO scale, etc.

\subsection{DESI Constraints}
% Expected sensitivity from DESI data
% Comparison with Planck and other surveys

\section{Axions and Axion-Like Particles (ALPs)}

Axions are pseudo-Nambu-Goldstone bosons arising from the Peccei-Quinn mechanism. Axion-like particles share similar properties but may arise from different symmetries.

\subsection{QCD Axion}
% Standard QCD axion as dark matter candidate
% Mass ranges and coupling constraints

\subsection{Ultralight Axions / Fuzzy Dark Matter}
% Effects on small-scale structure
% de Broglie wavelength suppression
% Impact on Lyman-$\alpha$ forest and small-scale power

\subsection{Axion-Photon Coupling}
% Effects on photon propagation
% Constraints from distance measurements

\subsection{DESI Sensitivity}
% How DESI constrains axion masses and couplings
% Complementarity with other experiments

\section{Cannibal Dark Matter}

Cannibal dark matter scenarios involve dark sector particles with strong self-interactions ($3 \to 2$ or higher-order processes) that remain in kinetic equilibrium after decoupling from the Standard Model.

\subsection{Cannibal Thermal History}
% Self-interaction processes
% Modified freeze-out dynamics
% Temperature evolution

\subsection{Cosmological Signatures}
% Effects on $N_{\text{eff}}$
% Modified matter-radiation equality
% Impact on structure formation

\subsection{Parameter Space}
% Cannibal coupling strengths
% Mass ranges
% Connection to dark sector models

\subsection{DESI Probes}
% How BAO and growth rate measurements constrain cannibal DM
% Synergies with CMB observations

\section{Decaying Dark Matter}

Dark matter decay can inject energy into the intergalactic medium and affect structure formation and cosmic expansion history.

\subsection{Decay Mechanisms}
% Two-body vs multi-body decays
% Visible vs invisible decay products
% Decay lifetimes and branching ratios

\subsection{Cosmological Effects}
% Energy injection and ionization history
% Modification of matter density evolution
% Effects on CMB and 21cm signal
% Impact on structure growth

\subsection{Constraints from Large-Scale Structure}
% How decaying DM affects power spectrum
% Velocity-dependent effects
% Scale-dependent signatures

\subsection{DESI Analysis}
% Constraining decay lifetime $\Gamma_{\chi}$
% Complementarity with CMB and X-ray observations
% Model-dependent vs model-independent constraints

\section{Interacting Dark Energy - Dark Matter}

\subsection{Coupling Models}
% Interaction terms in the cosmological equations
% Energy-momentum exchange

\subsection{Effects on Expansion History and Growth}
% Modified Hubble parameter
% Growth rate modifications


\section{Primordial Black Holes}

\subsection{PBH as Dark Matter}
% Mass windows
% Formation mechanisms

\subsection{Effects on Structure Formation}
% Extended vs point-like nature at different scales
% Clustering properties

\section{Analysis Strategy}

\subsection{Data and Observables}
% DESI DR1/DR2 specifications
% BAO measurements, RSD, power spectrum
% Cross-correlations

\subsection{Theoretical Modeling}
% Modified Boltzmann codes (CLASS, CAMB)
% Perturbation theory
% N-body simulations for non-linear regime

\subsection{Statistical Methods}
% MCMC analysis
% Nested sampling
% Model comparison (Bayesian evidence)

\subsection{Systematic Uncertainties}
% Observational systematics
% Theoretical uncertainties
% Degeneracies between parameters

\section{Expected Results and Forecasts}

\subsection{Individual Model Constraints}
% Forecast sensitivity for each BSM scenario

\subsection{Combined Constraints}
% Multiple BSM effects simultaneously
% Parameter degeneracies

\subsection{Comparison with Current Bounds}
% Improvement over Planck, SDSS, etc.

\section{Conclusions and Outlook}

\subsection{Summary}
% Recap of BSM scenarios and DESI sensitivity

\subsection{Future Prospects}
% DESI Y5 full survey
% Combination with Euclid, LSST, CMB-S4
% Next-generation experiments

\section*{Acknowledgments}
% Funding agencies, collaborations, etc.

\bibliographystyle{JHEP}
\bibliography{references}

\end{document}
